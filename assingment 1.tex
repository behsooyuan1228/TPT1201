\documentclass[12pt]{article}
\usepackage{mathptmx}
\usepackage{times}
\usepackage{url}
\usepackage[margin=0.9in,footskip=0.25in]{geometry}

\usepackage{array}

\title{Image Recognition Technique of Road Tax sticker in Malaysia}

\author{Beh Soo Yuan 1121116550\\Dorothea A/P Vincent 1121117087}

\begin{document}

\maketitle
\abstract{The inspection of car Road Tax information in Malaysia has been difficult as the employment of recognition system is limited due to the great incline of transportation systems. In order to combat this difficulty, this paper suggested a new image recognition technique known as Road Tax Recognition (RTR) system to observe the vehicle by automatically capturing and extracting the Road Tax sticker image based on Neural Network. RTR is a type of new image processing technology to identify Road Tax sticker as well as license plate of the vehicles in Malaysia, besides associating the accuracy of the matched information. With this new implementation, police will not have to check on the expiry date of the Road Tax sticker and match it with the car plate number manually. Furthermore, the method proposed by the article is widespread in terms of localization and recognition of various environmental and lighting conditions based on Neural Network techniques.}

\section{Problem solved}
Since the employment of recognition system in Malaysia is limited to the vehicle plate number, it means that the system is unable to detect Road Tax stickers. Because of it, the police have to observe the expiry date of the Road Tax sticker and also have to match it with the car plate number manually. This article proposes a new image recognition technique to monitor the vehicle by automatically capturing and extracting the road tax sticker image based on Neural Network. With the development of vehicle Road Tax Recognition (RTR) system, vehicles in Malaysia can be monitor in an easy and effective way. Thus will result a great efficiency for vehicle monitoring system in Malaysia.\cite{6516388}

\section{Related work/Prior research}
From this research, it will result in greater efficiency for vehicle monitoring system at Toll Gates in Malaysia and it will also strengthen the security of the area and traffic installation by verifying the details of the road tax sticker.

License Plate Recognition is being introduced into the industry at an increasing rate for many purposes. Research in the area of LPR has followed several approaches. In the paper of Yao Yuan, Wu xiao-li (2010) and Li Fuliang, Gao Shuangxi (2010) was solving the plate character recognition based on Neural Network.\cite{bpnn}\cite{digitalImg} Rodolfo Zunino, Stefano Rovetta (2000) have used Vector Quantization based image coding to obtain image compression for archival propose and to support the location process.\cite{vq} Several researchers addressed the problem of random noise disturbances due to the distance and lightning. So in Li Fuliang, Gao Shuangxi (2010) studied this problem can be improve by using back- propagation network. 

The shortcoming found in the prior literature of the selected article addresses is only in the Li Fuliang, Gao Shuangxi (2010) studied. The studies are lack of graphical evidence to prove the result.The research paper of the (Image Recognition Technique of Road Tax sticker in Malaysia) have carried out few tests to evaluate the effectiveness of the technique chosen. Thus, proving that the methods used are effective based on the data and result gain from the test.\cite{bpnn}

The authors have performed a well cited. Thus, there are no any relevant articles that have not been cited by the authors


\section{Methodology}
The method used in the paper of RTR system has six stages, Image Acquisition, Regional location, Road Tax extraction, Road tax segmentation, information recognition and Recognition result phases.
The first step of the RTR system uses a high-resolution digital camera to acquire the road tax sticker image. It then extracts the foreground image of the vehicle then convert the camera's video signal to digital image signals. \cite{par}

Next, the localization of the Road Tax sticker extract the coordinates of the vehicle Road Tax sticker area from the vehicle image, and then identify the sticker characters. Next, the image will be filtered to remove any noise and the Road Tax Sticker position is located by scanning the image vertically. Lastly, the sticker is scanned horizontally by moving a square window from left to right while counting the number of clusters inside. A square that contains the greatest number of clusters is the final position.   

Then, extraction of road tax is used to convert the camera's video signal to digital image signals to be sent to the computer for processing. A border removal mechanism is performed followed by the approximation of Road Tax Sticker height  

In the road tax segmentation step, character height is used to find out one essential component among the set of related component in the road tax sticker. For a better result, Character Mending Procedure is introduced.

In Character Recognition step, The Neural Network with Back Propagation Artificial (BP) is used to minimize the error sum squares of the network. Neural Network with Learning Vector Quantization (LVQ) is used to supervised learning to form classification. This two method are used, to ensure the best accuracy rate along with enhanced recognition speed by Neural Network.\cite{location}\cite{bpnn}

The method that has been used is suitable and appropriate in order to get the result they wanted. The authors have mostly used the research paper of Vehicle License plate recognition system as their reference. The most important steps in this RTR system is the character recognition, so in the research paper, the authors have used Neural Network with Back Propagation Artificial (BP) method and Neural Network with Learning Vector Quantization (LVQ) method to do character recognition.  

The possible problem that had been identified is when road tax area is not clear then the extraction becomes difficult. The foreground design with the characters in the road tax sticker is considered as a noise component. Besides, only the characters components within a narrow range of this base condition are considered as number sticker else are considered disposed. Improper illuminations can cause the character breakage.

Authors have carried out few test to evaluate the effectiveness of the technique by performing the Efficiency Test of Road Tax localization process and in the test they have gained 70\% in the trials. The second test they have performed is the Efficiency Test of character segmentation process. The test has gained 67\% in the trials. The last test is Efficiency Test of character recognition process and it gained 64\% in the trials. All test are using on the same data set, which consist of 150 pictures. 

\section{Claimed Contributions}
As a result from the methodology that has been implemented, the significant result obtained from the research paper is that the invention of a system that is able to identify license and Road Tax numbers precisely at toll gates. Three tests are performed namely the Efficiency Test of Road Tax localization process, Efficiency Test of character segmentation process, and Efficiency Test of character recognition process in order to assess the effectiveness of the proposed technique. All three trials showed high percentage of accuracy from the various Road Tax types of data sets taken from Toll gates lighting. Thus, having these positive results, this clarifies the plate number and expiry date that is written in the road tax stickers and matched with license plate.
\section{Conclusion}
The overall flow of the research is focused on the localization and recognition of different Road Tax sticker images under many different environmental and lighting conditions based on Neural Network techniques. Not only that, feature extraction approaches was also been proposed. It is found that the simulation accuracy is shown to be high and this is due to the fact that the fan beam feature extraction method consists of more features to train the Neural Network technique. The main feature that been explored through the research is on the extraction and recognition methods. Therefore, manual method can be eliminated by implementing this particular technique in monitoring vehicles automatically by capturing and extracting the Road Tax sticker image. 
\section{What did you learn? And possible extension/Future work}
This technique that been suggested is although universal, the average quality of all pictures are limited by three elements: resolution 640x480 points, 24-bit or 8-bit colour space and of JPEG compression. As for future extension works, more number of fonts can be used for the network to improve the character recognition. Besides that, as the images are limited to one format of JPEG, more research can be done to enable the efficiency of the system to be broad into other formats like PNG and BMP. 
As this system is set at toll gates, where it captures the image of Road Tax when the vehicle is at stationary, expanding the system features so that it is able to detect and capture the image at any speed will be a great improvisation of the system. Therefore, this enables the system to be located not only at toll gates and also in other areas along highways.   

\pagebreak
\bibliographystyle{plain}
\bibliography{refer}

\begin{tabular}{|m{6cm} |m{10cm}|}
\hline 
Paper Title & \\ 
\hline 
Author(s) & \\ 
\hline 
Abstract/Summary & \\ 
\hline 
Problem Solved & \\ 
\hline 
Claimed Contributions & \\ 
\hline 
Related Work & \\ 
\hline 
Methodology & \\ 
\hline 
Conclusions & \\ 
\hline 
What did you learn 
\newline (algorithms/experiments
details)?
\newline
\newline 
Possible extension/Future
work & \\ 
\hline 
References & \\ 
\hline 
\end{tabular} 

\end{document}